%%%%%%%%%%%%%%%%%%%%%%%%%%%%%%%%%%%%%%%%%%%%%%%%%%%%%%%%%%%%%%%%%%%%%
% LaTeX Template: Project Titlepage Modified (v 0.1) by rcx
%
% Original Source: http://www.howtotex.com
% Date: May 17 2018
% 
% This is a title page template which be used for articles & reports.
% 
% This is the modified version of the original Latex template from
% aforementioned website.
% 
%%%%%%%%%%%%%%%%%%%%%%%%%%%%%%%%%%%%%%%%%%%%%%%%%%%%%%%%%%%%%%%%%%%%%%

\documentclass[12pt]{report}
%\usepackage{csvsimple}
%% Sets page size and margins
\usepackage[a4paper,top=3cm,bottom=2cm,left=3cm,right=3cm,marginparwidth=2.0cm]{geometry}
% \usepackage[myheadings]{fullpage}
\usepackage{fancyhdr}
\usepackage{framed}
\usepackage{lastpage}
\usepackage{graphicx}
\graphicspath{{images/}}
\usepackage{wrapfig, subcaption, setspace, booktabs}
\usepackage[T1]{fontenc}
\usepackage[font=small, labelfont=bf]{caption}
\usepackage{fourier}
\usepackage[protrusion=true, expansion=true]{microtype}
\usepackage[english]{babel}
\usepackage{sectsty}
\usepackage{listings}
\lstset{frame=tb,
  breaklines=true,
  showstringspaces=false,
  columns=flexible,
  numbers=none,
  commentstyle=\color{dkgreen},
  stringstyle=\color{mauve},
  tabsize=1,
}
\usepackage{rotating}
\usepackage{url, lipsum}
\usepackage{verbatim}
\newcommand{\HRule}[1]{\rule{\linewidth}{#1}}
\onehalfspacing
\setcounter{tocdepth}{5}
\setcounter{secnumdepth}{5}
\usepackage{alltt}
\usepackage[title]{appendix}
\usepackage[utf8]{inputenc}
\usepackage[english]{babel}
\usepackage{color, colortbl}
\definecolor{LightCyan}{rgb}{0.88,1,1}
\usepackage{minted}
\usepackage{hyperref}
\hypersetup{
    colorlinks=true,
    linkcolor=blue,
    filecolor=magenta,      
    urlcolor=blue,
}
\usepackage{todonotes}

%-------------------------------------------------------------------------------
% HEADER & FOOTER
%-------------------------------------------------------------------------------
\pagestyle{fancy}
\fancyhf{}
\setlength\headheight{15pt}
\fancyhead[L]{Student Name}
\fancyhead[R]{University of Toronto}
\fancyhead[R]{Röst Lab Project}
\fancyfoot[R]{Page \thepage\ of \pageref{LastPage}}
%-------------------------------------------------------------------------------
% TITLE PAGE
%-------------------------------------------------------------------------------

\begin{document}

\title{ \textsc{Röst Lab}
		\\ [2.0cm]
		\HRule{0.5pt} \\
		\LARGE \textbf{{Lab Journal}}
		\HRule{2pt} \\ [0.5cm]
		\normalsize May 17, 2018\vspace*{5\baselineskip}}

\date{}

\author{
		Student Name \\ 
		Donnelly Centre\\
		University of Toronto \\
		Hannes Röst Lab }

\maketitle
\tableofcontents
\newpage

%-------------------------------------------------------------------------------
% Section title formatting
\sectionfont{\scshape}
\renewcommand\thesection{\arabic{section}.}
\renewcommand\thesubsection{\thesection\arabic{subsection}}
\sectionfont{\fontsize{8}{10}}
%-------------------------------------------------------------------------------

%-------------------------------------------------------------------------------
% BODY
%-------------------------------------------------------------------------------

\section{Week 1: January}
\subsection{January 1}

There is a theory which states that if ever anyone discovers exactly what the Universe is for and why it is here, it will instantly disappear and be replaced by something even more bizarre and inexplicable.
There is another theory which states that this has already happened. \todo{important note to remember this later}

\subsection{January 2}

The density profile follows an $r^{-3/2}$ power law. To avoid a singularity at the center, an interpolation is done over a radius. This inner radius is defined in the parameter file. It should follow the prescription of a singular isothermal sphere (see Binney \& Tremaine p.305), which is also the definition of the King radius:
\begin{equation}
r_0 \equiv \sqrt{\frac{9\sigma^2}{4\pi G\rho_0}}
\end{equation}
where $\sigma$ is the velocity dispersion and could be estimated as $\sigma = \mathcal{M} c_s$, where $c_s = \sqrt{\gamma P/\rho} = \sqrt{\gamma k_B T / \mu}$ is the sound speed.

The isothermal sound speed in our simulation was estimated
\begin{equation}
c_s = \sqrt{\frac{k_b T}{\mu m_p}}
\end{equation}
I'm unsure why a factor of $\gamma$ was not included. For 30 K, this gives a sound speed of about 34000 cm/s or 0.34 km/s. At a Mach number of 5, this gives a supersonic dispersion of $\sigma$ = 1.7 km/s


\subsection{January 3}

A small bash script:

\begin{minted}[fontsize=\footnotesize]{bash}
#!/bin/sh
#PeptideProphet analysis
for f in *.mzXML;
do
  echo $f  
done
\end{minted}

A small Python script:

\begin{minted}[fontsize=\footnotesize]{python}
import pyopenms
exp = pyopenms.MSExperiment()
pyopenms.MzMLFile().store("testfile.mzML", exp)
\end{minted}

\end{document}
